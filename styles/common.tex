
% -------------------------------------------------------
%  Common Styles and Formattings
% -------------------------------------------------------


\usepackage{amssymb,amsmath}
\usepackage{physics}
\usepackage[colorlinks,linkcolor=blue,citecolor=blue]{hyperref}
\usepackage[usenames,dvipsnames]{pstricks}
\usepackage{graphicx,subfigure,wrapfig}
\usepackage{geometry,fancyhdr}
\usepackage[mathscr]{euscript}
\usepackage{multicol}

\usepackage{algorithmicx,algorithm}

\usepackage[extrafootnotefeatures]{xepersian}
\usepackage[noend]{algpseudocode}

\usepackage[bottom]{footmisc}
%\addtolength{\skip\footins}{1pc}

\usepackage{setspace}

% -------------------- Page Layout --------------------


\newgeometry{top=3.5cm,bottom=3.5cm,left=2.5cm,right=3cm,headheight=25pt}

\renewcommand{\baselinestretch}{1.4}
\linespread{1.6}
\setlength{\parskip}{0.45em}

\fancyhf{}
%\rhead{\leftmark}
%\lhead{\thepage}
\fancyhead[R]{\rightmark}
\fancyhead[L]{\thepage}

\usepackage{perpage}
\MakePerPage{footnote}
% -------------------- Fonts --------------------

\settextfont[
Scale=1.09,
Extension=.ttf, 
Path=styles/fonts/,
BoldFont=XB NiloofarBd,
ItalicFont=XB NiloofarIt,
BoldItalicFont=XB NiloofarBdIt
]{XB Niloofar}
%%% table
\usepackage{array, tabularx, makecell, booktabs, multirow}
\usepackage[table]{xcolor}
%\setdigitfont[
%Scale=1.09,
%Extension=.ttf, 
%Path=styles/fonts/,
%BoldFont=XB NiloofarBd,
%ItalicFont=XB NiloofarIt,
%BoldItalicFont=XB NiloofarBdIt
%]{XB Niloofar}


\defpersianfont\sayeh[
Scale=1,
Path=styles/fonts/
]{XB Kayhan Pook}

%_____________________python code_________________

\usepackage{minted}
\usepackage{xcolor}


% -------------------- Styles --------------------


\SepMark{-}
\renewcommand{\labelitemi}{$\small\bullet$}



% -------------------- Environments --------------------


\newtheorem{قضیه}{قضیه‌ی}[chapter]
\newtheorem{لم}[قضیه]{لم}
\newtheorem{ادعا}[قضیه]{ادعای}
\newtheorem{مشاهده}[قضیه]{مشاهده‌ی}
\newtheorem{نتیجه}[قضیه]{نتیجه‌ی}
\newtheorem{مسئله}{مسئله‌ی}[chapter]
\newtheorem{تعریف}{تعریف}[chapter]
\newtheorem{مثال}{مثال}[chapter]


\newenvironment{اثبات}
	{\begin{trivlist}\item[\hskip\labelsep{\em اثبات.}]}
	{\leavevmode\unskip\nobreak\quad\hspace*{\fill}{\ensuremath{{\square}}}\end{trivlist}}

\newenvironment{alg}[2]
	{\begin{latin}\settextfont[Scale=1.0]{Times New Roman}
	\begin{algorithm}[t]\caption{#1}\label{algo:#2}\vspace{0.2em}\begin{algorithmic}[1]}
	{\end{algorithmic}\vspace{0.2em}\end{algorithm}\end{latin}}



% python 

% -------------------- Commands --------------------


\newcommand{\IN}{\ensuremath{\mathbb{N}}} 
\newcommand{\IZ}{\ensuremath{\mathbb{Z}}} 
\newcommand{\IQ}{\ensuremath{\mathbb{Q}}} 
\newcommand{\IR}{\ensuremath{\mathbb{R}}} 
\newcommand{\IC}{\ensuremath{\mathbb{C}}} 

\newcommand{\set}[1]{\left\{ #1 \right\}}
\newcommand{\seq}[1]{\left< #1 \right>}
\newcommand{\ceil}[1]{\left\lceil{#1}\right\rceil}
\newcommand{\floor}[1]{\left\lfloor{#1}\right\rfloor}
\newcommand{\card}[1]{\left|{#1}\right|}
\newcommand{\setcomp}[1]{\overline{#1}}
\newcommand{\provided}{\,:\,}
\newcommand{\divs}{\mid}
\newcommand{\ndivs}{\nmid}
\newcommand{\iequiv}[1]{\,\overset{#1}{\equiv}\,}
\newcommand{\imod}[1]{\allowbreak\mkern5mu(#1\,\,\text{پیمانه‌ی})}

\newcommand{\poly}{\mathop{\mathrm{poly}}}
\newcommand{\polylog}{\mathop{\mathrm{polylog}}}
\newcommand{\eps}{\varepsilon}

\newcommand{\lee}{\leqslant}
\newcommand{\gee}{\geqslant}
\renewcommand{\leq}{\lee}
\renewcommand{\le}{\lee}
\renewcommand{\geq}{\gee}
\renewcommand{\ge}{\gee}

\newcommand{\مهم}[1]{\textbf{#1}}
\newcommand{\برچسب}{\label}

\newcommand{\REM}[1]{}
\newcommand{\حذف}{\REM}
\newcommand{\لر}{\lr}
\newcommand{\کد}[1]{\lr{\tt #1}}
\newcommand{\ft}[1]{\LTRfootnote{\lr{\ #1}}}


% ------------------------------ Images and Figures --------------------------

\graphicspath{{figs/}}
\setlength{\intextsep}{0pt}  % for float boxes
\renewcommand{\psscalebox}[1]{}  % for LaTeX Draw

\newcommand{\floatbox}[2]
	{\begin{wrapfigure}{l}{#1}
	\centering #2 \end{wrapfigure}}

\newcommand{\centerfig}[2]
	{\centering\scalebox{#2}{\input{figs/#1}}}

\newcommand{\fig}[3]
	{\floatbox{#3}{\centerfig{#1}{#2}}}

\newcommand{\centerimg}[2]
	{
		\vspace{1em}
		\begin{center}\includegraphics[width=#2]{figs/#1}\end{center}
		\vspace{-1em}
	}

\NewDocumentCommand{\img}{m m o}
	{\begin{wrapfigure}{l}{\IfValueTF{#3}{#3}{#2}}
	\centering\includegraphics[width=#2]{figs/#1}\end{wrapfigure}}

%--------------------------------------------------------------------------

\newcommand{\icol}[1]{% inline column vector
	\left(\begin{smallmatrix}#1\end{smallmatrix}\right)%
}

\newcommand{\irow}[1]{% inline row vector
	\begin{smallmatrix}(#1)\end{smallmatrix}%
}
